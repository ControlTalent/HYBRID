\section{Introduction}
% High-level HYBRID description
One of the goals of the HYBRID modeling and simulation project is to assess the economic viability of hybrid systems in a market that contains renewable energy sources like wind. The hybrid system would be a nuclear reactor that not only generates electricity, but also provides heat to another plant that produces by-products, like hydrogen or desalinated water. The idea is that the possibility of selling heat to a heat user absorbs (at least part of) the market volatility introduced by the renewable energy sources.

The system that is studied is modular and made of an assembly of components. For example, a system could contain a hybrid nuclear reactor, a gas turbine, a battery and some renewables. This system would correspond to the size of a balance area, but in theory any size of system is imaginable. The system is modeled in the ‘Modelica/Dymola’ language.
To assess the economics of the system, an optimization procedure is varying different parameters of the system and tries to find the minimal cost of electricity production.

\subsection{Modelica Models}
Idaho National Laboratory (INL) has been developing the NHES package, a library of high-fidelity process models in the commercial Modelica language platform Dymola since early 2013 \cite{2017Report}, \cite{2016HTSE}, \cite{2018ThermalStorage}, \cite{2019NuScaleM4}. The Modelica language is a non-proprietary, object oriented, equation-based language that is used to conveniently model complex, physical systems. Modelica is an inherently time-dependent modeling language that allows the swift interconnection of independently developed models. Being an equation-based modeling language that employs differential algebraic equation (DAE) solvers, users can focus on the physics of the problem rather than the solving technique used, allowing faster model generation and ultimately analysis. This feature alongside system flexibility has led to the widespread use of the Modelica language across industry for commercial applications. System interconnectivity and the ability to quickly develop novel control strategies while still encompassing overall system physics is why INL has chosen to develop the Integrated Energy Systems (IES) framework in the Modelica language.

\subsection{Individual Components}
The current version of the NHES library employs both third party components from the Modelica Standard Library \cite{ModelicaAssociation} and TRANSFORM \cite{TRANSFORM} and components developed internal to the project for specific subsystems. For example, the NHES library contains a large variety of models for the development of a high-temperature steam electrolysis plant, a gas turbine, a basic Rankine cycle balance of plant, and a light water nuclear reactor. Components included in the library that support the development of these systems include 1-D pipes, pressurizers, condensers, turbines (steam and gas), heat exchangers, a simple logic-based battery, a nuclear fuel subchannel, etc. Third party models include numerous additional models including source/sink components (e.g., fluid boundary conditions), additional heat exchanger models, logical components for control system development, multi-body components, additional supporting functions (e.g., LAPACK, interpolation, smoothing), etc. Please see the specific libraries for additional information. 

\subsection{Hybrid Requirements}


The repository itself can be found here: \url{https://hpcgitlab.inl.gov/hybrid/hybrid}
 
\textbf{Software requirements are as follows:}

\begin{enumerate}
\item Commercial Modelica platform Dymola -- \url{https://www.3ds.com/products-services/catia/products/dymola/latest-release/}.
\item Risk Analysis and Virtual ENviroment (RAVEN) -- \url{https://raven.inl.gov/SitePages/Software%20Infrastructure.aspx}
\item Python 3 -- \url{https://docs.conda.io/en/latest/miniconda.html}
\item Microsoft Visual Studio Community Edition. -- \url{https://visualstudio.microsoft.com/downloads/}
\end{enumerate}

	

\textbf{Note}: Steps 3 and 4 can be accomplished by following the RAVEN installation instructions in step two. The installation procedure will be outlined below. 
All physical models are run within the Dymola simulation framework graphical user interface (GUI).  Background information on the Modelica as a language as well as good general guidance on coding practices can be found at the two references shown below. 
\begin{enumerate}
\item \url{https://webref.modelica.university/}
\item \url{https://mbe.modelica.university/}
\end{enumerate}

